\documentclass{article}
\usepackage[backend=bibtex, style=numeric]{biblatex}
\usepackage[dvipsnames]{xcolor}
\bibliography{MyCitations}
\title{Gaussian Processes}
\begin{document}
\maketitle

%###############
\section{Introduction}
%###############

%======================
\subsection{Basis Functions}
%======================
Basis functions and their combinations can be of the following 16 types, given in blue:
\begin{enumerate}
\item {\color{red}\textbf{Linear basis functions, linear combination}}: none
\item {\color{red}\textbf{Linear basis functions, non-linear combination}}: none
\item {\color{red}\textbf{Non-linear basis functions, linear combination}}
\begin{enumerate}
\item {\color{OliveGreen}\textbf{Fixed type}}
\begin{enumerate}
\item {\color{blue}\textbf{Fixed number}}: Ch 3 \& 4 (read pg 227): Polynomial Regression
\end{enumerate}
\end{enumerate}
\item {\color{red}\textbf{Non-linear basis functions, non-linear combination}}
\end{enumerate}

\cite{GaussProc_SimpleIntro}
\cite{BOOK_GMPL}

\printbibliography
\end{document}
