\documentclass{article}
\usepackage[backend=bibtex, style=numeric]{biblatex}
\usepackage[dvipsnames]{xcolor}
\bibliography{MyCitations}
\title{Gaussian Processes}
\begin{document}
\maketitle

%###############
\section{Introduction}
%###############

%======================
\subsection{Basis Functions}
%======================
We are not interested in linear basis functions.
The output function depends on non-linear basis functions and their combinations in the following manner:

\begin{enumerate}
\item {\color{red}\textbf{Linear combination}}
\begin{enumerate}
\item {\color{OliveGreen}\textbf{Fixed type}}
\begin{enumerate}
\item {\color{blue}\textbf{Fixed number}}: Polynomial Regression (Ch 3 \& 4, read pg 227) 
\item {\color{blue}\textbf{Variable number, depends on data}}: SVM (Ch 6, 7)
\end{enumerate}
\end{enumerate}
\item {\color{red}\textbf{Non-linear combination}}
\begin{enumerate}
\item {\color{OliveGreen}\textbf{Variable type}}
\begin{enumerate}
\item {\color{blue}\textbf{Fixed number}}: Neural Networks (Ch 5)
\end{enumerate}
\end{enumerate}
\end{enumerate}

\cite{GaussProc_SimpleIntro}
\cite{BOOK_GMPL}

\printbibliography
\end{document}
